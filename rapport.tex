\documentclass{article}

\usepackage[latin1]{inputenc}
\usepackage{color}
\usepackage{graphicx}
\usepackage{url}

\RequirePackage{hyperref} %% Clicka-proof links
\urlstyle{sf}

\usepackage{listings}

\addtolength{\hoffset}{-2cm}
\addtolength{\textwidth}{3cm} 

\title{Projet de réalité virtuelle: Construction en Kapla}
\author{Pierre Vittet, Guillamue Smaha}
\begin{document}

\chapter{Introduction}
\section{Présentation du projet}
    L'objectif du projet est de proposer un environnement 3D dans lequel un
    utilisateur peut positionner un ensemble de briquettes afin de créer des
    structures. A partir d'une table, l'utilisateur doit essayer d'obtenir une
    structure qui est la flèche la plus grande possible (c'est à dire qui soit
    éloigné autant que possible du bord de la table.
    Bien sur, chaque briquette est soumis à la force de gravité rendant plus
    complexe l'élaboration de structures stables.
\section{}
    Aprés avoir discuter avec notre encadrant, nous nous sommes données les objectifs suivants:
    \begin{itemize}
        \item Un menus permettant de gérer différents niveaux de difficulté.
        Dans un premier temps la seul différence viendra du nombre de
        briquettes disponibles, mais il est également possible de travailler
        sur le poid ou la taille des briquettes.
        \item Saisis des objets et déplacement des briquettes via la souris
        \item Possibilité de revenir à un état précédant ou sucesseur du jeu:
        Aprés chaque placement de briquette, l'état du jeu est mémorisé et on
        permet d'y revenir.
        
    \end{itemize}
\chapter{Choix techniques}
    \section{}
    L'encadrant de projet nous à laissé libre de choisir les technologies et
    les outils que l'on souhaitait utilisé pour le projet. Nos choix techniques
    ont été pris de manière à offrir un logiciel facilement utilisable sur
    différentes plateformes mais également en vue d'aller aussi loins que
    possible dans le projet en considérant le temps imparti. Nous avons fait le
    choix de réutiliser autant que possible les outils que nous connaissions
    déja. Cela nous à permis d'avoir une vue d'ensemble et une maitrise que
    nous n'aurions pas eu autrement.

    \section{Liste des outils}
    \begin{itemize}
        \item Ogre: Moteur 3D open source (http://www.ogre3d.org/) supportant aussi bien OpenGL que Direct3D. 
        \item CEGUI: Crazy Eddie's GUI System (http://www.cegui.org.uk) fournissant des outils de créations de menus et de fenetre dans un environnement 3D.
        \item Bullet: Librairie de simulation de la physique http://bulletphysics.org/
    \end{itemize}
