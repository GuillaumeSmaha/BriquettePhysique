\documentclass[frenchb,twoside]{EPURapport}

%\usepackage{listings}

%\renewcommand{\lstlistlistingname}{Liste des codes}
%\renewcommand{\lstlistingname}{Code}

%\addextratables{%
%	\lstlistoflistings
%}

%\swapAuthorsAndSupervisors

\RequirePackage{listings}

% le compteur de references
\makeatletter
\newcounter{reference}%
\renewcommand \thereference
 {\@arabic\c@reference}
\makeatother

\newenvironment{bibliographie}{\begin{small}\begin{enumerate}}{\end{enumerate}\end{small}}


\newcommand{\biblioentry}[2]{%
	\refstepcounter{reference}
	\label{#1}
	\ifthenelse{\equal{#2}{}}{%	
		\item[{[\ref{#1}]}] \xspace
	}{%
		\item[{[#2]}] \xspace
	}
}

\newcommand{\biblioentryFull}[5]{%
	\refstepcounter{reference}
	\label{#1}
	\ifthenelse{\equal{#2}{}}{%	
		\item[{[\ref{#1}]}] \xspace #3, \og \textit{#4} \fg, \url{#5}
	}{%
		\ifthenelse{\equal{#5}{}}{%	
			\item[{[#2]}] \xspace \xspace #3, \og \textit{#4} \fg
		}{%
			\item[{[#2]}] \xspace \xspace #3, \og \textit{#4} \fg, \url{#5}
		}
	}
}

\newcommand{\refMot}[1]{%
	\up{[\ref{#1}]}
}

\newcommand{\labelMot}[1]{%
	\refstepcounter{reference}
	\label{#1}
}


\newcommand{\lstinputlistingSmall}[1]{%
	\begin{scriptsize}
		\lstinputlisting{#1}
	\end{scriptsize}
}



%\usepackage[utf8]{inputenc}
\usepackage[utf8x]{inputenc}
\usepackage{ucs}
\usepackage{eurosym}
\usepackage{graphicx}
\usepackage{url}
\usepackage{pdfpages}
\usepackage[makeindex]{glossaries}
%\usepackage{makeidx}
%\usepackage{showidx}

\urlstyle{sf}



%%%%%%%%%%%%Commande personnalisée
\newcommand\motClef[1]{\textbf{\gls{#1}}}
\newcommand{\exemple}{\textbf}
\newcommand{\info}{\textbf}


%\makeindex
\makeglossaries
\thedocument{Projet de r\'{e}alit\'{e} virtuelle}{Construction en Kapla}{Construction en Kapla}

\grade{Département Informatique\\ 5\ieme{} année\\ 2010 - 2011}

\authors{%
	\category{Étudiants}{%
		\name{Guillaume Smaha} \mail{guillaume.smaha@etu.univ-tours.fr}
		\name{Pierre Vittet} \mail{pierre.vittet@etu.univ-tours.fr}
	}
	\details{DI5 2010 - 2011}
}

\supervisors{%
	\category{Encadrant}{%
		\name{Sebastien Aupetit} \mail{aupetit@univ-tours.fr}
		\name{Emmanuel Néron} \mail{emmanuel.neron@univ-tours.fr}
	}
	\details{Université François-Rabelais, Tours}
}

\abstracts{Ce rapport présentera notre projet de construction de Kapla. L'objectif de ce projet était de développer une simulation physique d'un jeu de Kapla. Nous y expliquerons les différentes techniques utilisées (signaux, déplacements, ...) pour obtenir le résultat le plus réaliste possible.}
{Ogre3D, Bullet, Kapla, simulation}
{This report will describe our project of construction Kapla. The main goal of this project was to develop a physic simulation of a Kapla game. We will describe the different techniques (signals, movement, …) used to obtain the most realistic result.}
{Ogre3D, Bullet, Kapla, simulation}


\begin{document}


\chapter{Introduction}

\section{Présentation du projet}
    L'objectif du projet est de proposer un environnement 3D dans lequel un
    utilisateur peut positionner un ensemble de briquettes afin de créer des
    structures. A partir d'une table, l'utilisateur doit essayer d'obtenir une
    structure qui est la flèche la plus grande possible (c'est à dire qui soit
    éloigné autant que possible du bord de la table.
    Bien sur, chaque briquette est soumis à la force de gravité rendant plus
    complexe l'élaboration de structures stables.

\section{Objectifs}


\section{}
    Aprés avoir discuter avec notre encadrant, nous nous sommes données les objectifs suivants:
    \begin{itemize}
        \item Un menus permettant de gérer différents niveaux de difficulté.
        Dans un premier temps la seul différence viendra du nombre de
        briquettes disponibles, mais il est également possible de travailler
        sur le poid ou la taille des briquettes.
        \item Saisis des objets et déplacement des briquettes via la souris
        \item Possibilité de revenir à un état précédant ou sucesseur du jeu:
        Aprés chaque placement de briquette, l'état du jeu est mémorisé et on
        permet d'y revenir.
        
    \end{itemize}
   
    
    
\chapter{Choix techniques}
    \section{Contexte}
    L'encadrant de projet nous à laissé libre de choisir les technologies et
    les outils que l'on souhaitait utilisé pour le projet. Nos choix techniques
    ont été pris de manière à offrir un logiciel facilement utilisable sur
    différentes plateformes mais également en vue d'aller aussi loins que
    possible dans le projet en considérant le temps imparti. Nous avons fait le
    choix de réutiliser autant que possible les outils que nous connaissions
    déja. Cela nous à permis d'avoir une vue d'ensemble et une maitrise que
    nous n'aurions pas eu autrement.

    \section{Liste des outils}
    \begin{itemize}
        \item Ogre\refMot{bib:librairie_ogre}: Moteur 3D open source (http://www.ogre3d.org/) supportant aussi bien OpenGL que Direct3D. 
        \item CEGUI\refMot{bib:librairie_cegui}: Crazy Eddie's GUI System (http://www.cegui.org.uk) fournissant des outils de créations de menus et de fenetre dans un environnement 3D.
        \item Bullet\refMot{bib:librairie_bullet}: Librairie de simulation de la physique http://bulletphysics.org/
    \end{itemize}


\chapter{Conclusion}




\chapter{Bibliographie}

\begin{bibliographie}	
	\biblioentry{bib:ogre_quaternion}{} Ogre Wiki, \textit{Utilisation des quaternions dans Ogre}, \url{http://www.ogre3d.org/tikiwiki/Quaternion+and+Rotation+Primer}
	\biblioentry{bib:librairie_ogre}{} Ogre Documentation, \textit{Documentation Doxygen d'Ogre}, \url{http://www.ogre3d.org/docs/api/html/index.html}
	\biblioentry{bib:librairie_cegui}{} CEGUI Wiki, \textit{Utilisation de CEGUI}, \url{http://www.cegui.org.uk/wiki/index.php/Main_Page}
	\biblioentry{bib:librairie_bullet}{} Bullet Wiki, \textit{Utilisation de Bullet}, \url{http://bulletphysics.org/wordpress/}
	\biblioentry{bib:librairie_ogrebullet}{} Ogre Wiki, \textit{Tutorial d'utilisation d'OgreBullet}, \url{http://www.ogre3d.org/tikiwiki/OgreBullet}
	\biblioentry{bib:signaux_qt}{} Site du Zéro, \textit{Signaux avec QT}, \url{http://www.siteduzero.com/tutoriel-3-11268-les-signaux-et-les-slots.html}
	\end{bibliographie}	


\end{document}
