\documentclass[frenchb,twoside]{EPURapport}

%\usepackage{listings}

%\renewcommand{\lstlistlistingname}{Liste des codes}
%\renewcommand{\lstlistingname}{Code}

%\addextratables{%
%	\lstlistoflistings
%}

%\swapAuthorsAndSupervisors

\RequirePackage{listings}

% le compteur de references
\makeatletter
\newcounter{reference}%
\renewcommand \thereference
 {\@arabic\c@reference}
\makeatother

\newenvironment{bibliographie}{\begin{small}\begin{enumerate}}{\end{enumerate}\end{small}}


\newcommand{\biblioentry}[2]{%
	\refstepcounter{reference}
	\label{#1}
	\ifthenelse{\equal{#2}{}}{%	
		\item[{[\ref{#1}]}] \xspace
	}{%
		\item[{[#2]}] \xspace
	}
}

\newcommand{\biblioentryFull}[5]{%
	\refstepcounter{reference}
	\label{#1}
	\ifthenelse{\equal{#2}{}}{%	
		\item[{[\ref{#1}]}] \xspace #3, \og \textit{#4} \fg, \url{#5}
	}{%
		\ifthenelse{\equal{#5}{}}{%	
			\item[{[#2]}] \xspace \xspace #3, \og \textit{#4} \fg
		}{%
			\item[{[#2]}] \xspace \xspace #3, \og \textit{#4} \fg, \url{#5}
		}
	}
}

\newcommand{\refMot}[1]{%
	\up{[\ref{#1}]}
}

\newcommand{\labelMot}[1]{%
	\refstepcounter{reference}
	\label{#1}
}


\newcommand{\lstinputlistingSmall}[1]{%
	\begin{scriptsize}
		\lstinputlisting{#1}
	\end{scriptsize}
}



%~ \usepackage[utf8]{inputenc}
\usepackage[utf8x]{inputenc}
\usepackage{ucs}
\usepackage{eurosym}
\usepackage{graphicx}
\usepackage{url}
\usepackage{pdfpages}
\usepackage[makeindex]{glossaries}
%\usepackage{makeidx}
%\usepackage{showidx}

\urlstyle{sf}

\lstset{language=C++}  


%%%%%%%%%%%%Commande personnalisée
\newcommand\motClef[1]{\textbf{\gls{#1}}}
\newcommand{\exemple}{\textbf}
\newcommand{\info}{\textbf}


%\makeindex
\makeglossaries
\thedocument{Projet de r\'{e}alit\'{e} virtuelle}{Construction en Kapla}{Construction en Kapla}

\grade{Département Informatique\\ 5\ieme{} année\\ 2010 - 2011}

\authors{%
	\category{Étudiants}{%
		\name{Guillaume Smaha} \mail{guillaume.smaha@etu.univ-tours.fr}
		\name{Pierre Vittet} \mail{pierre.vittet@etu.univ-tours.fr}
	}
	\details{DI5 2010 - 2011}
}

\supervisors{%
	\category{Encadrant}{%
		\name{Sebastien Aupetit} \mail{aupetit@univ-tours.fr}
		\name{Emmanuel Néron} \mail{emmanuel.neron@univ-tours.fr}
	}
	\details{Université François-Rabelais, Tours}
}

\abstracts{Ce rapport présentera notre projet de logiciel 3D de construction de
Kapla. L'objectif de ce projet était de développer une simulation physique d'un
jeu de Kapla. Nous y expliquerons les différentes techniques utilisées
(signaux, déplacements, ...) pour obtenir le résultat le plus réaliste
possible.}
{Ogre3D, Bullet, Kapla, simulation}
{This report will describe our project of a 3D software for building with
Kapla. The main goal of this project was to develop a physic simulation of a
Kapla game. We will describe the different techniques (signals, movement, …)
used to obtain the most realistic result.}
{Ogre3D, Bullet, Kapla, simulation}


\begin{document}


\chapter{Introduction}

\section{Présentation du projet}

	Ce projet se place dans le cadre de l'option Réalité Virtuelle pour les
	élèves de 5\up{ième} année de l'école Polytech'Tours.
	
	\
	
	Ce projet a été proposé par Emmanuel Néron afin de répondre à la demande
	de concrétiser le problème d'ordonnancement pour un cas particulier du jeu
	de "Kapla". Le problème d'ordonnancement est, pour un nombre donné de
	briquettes, de trouver la meilleure architecture permettant d'avoir une
	structure stable et d'être éloignée le plus possible du support de départ.

\section{Objectifs}
    L'objectif du projet est de proposer un environnement 3D dans lequel un
    utilisateur peut positionner un ensemble de briquettes afin de créer des
    structures. A partir d'une table, l'utilisateur doit essayer d'obtenir une
    structure qui ait la flèche la plus grande possible (c'est à dire qui soit
    éloignée autant que possible du bord de la table).
    Bien sûr, chaque briquette est soumise à la force de gravité rendant plus
    complexe l'élaboration de structures stables.
    Ce document décrit les technologies utilisées ainsi que le fonctionnement du
    logiciel.
    
    \begin{figure}[h]
		\centering
        \includegraphics[width=13cm]{images/jeux.png}
        \caption{\label{fig:jeux}Un exemple typique de l'application en cours de fonctionnment}
    \end{figure}

\section{Détail des attentes}
    Après avoir discuté avec notre encadrant, nous nous sommes donnés les objectifs suivants:
    \begin{itemize}
        \item Un menu permettant de gérer différents niveaux de difficulté.
        Dans un premier temps la seul différence viendra du nombre de
        briquettes disponibles, mais il est également possible de travailler
        sur le poids ou la taille des briquettes.
        \item Saisie des objets et déplacement des briquettes via la souris
        \item Possibilité de revenir à un état précédant ou sucesseur du jeu:
        Après chaque placement de briquette, l'état du jeu est mémorisé et on
        permet d'y revenir.
        \item Les briquettes ne doivent pas pouvoir être posées n'importe où
        mais seulement le long d'un axe. De même, leur orientation est
        vérouillée, l'utilisateur ne pouvant pas placer ces briquettes en
        travers. Cela permet de réduire la complexité du problème lorsque l'on
        recherche des solutions efficaces au problème.
        Le jeu n'est donc en réalité pas un jeu 3D mais 2D, seule la visualisation est en 3D.
        
    \end{itemize}
     \begin{figure}[h]
		\centering
        \includegraphics[width=16cm]{images/zoom_briquette.png}
        \caption{\label{fig:zoom_briquette}Les briquettes sont toutes orientées
        de la même façon suivant le même axe.}
    \end{figure}

   
    
    
\chapter{Choix techniques}
    \section{Contexte}
    L'encadrant de projet nous a laissé la possibilité de choisir les technologies et
    les outils que l'on souhaitait utiliser pour le projet. Nos choix techniques
    ont été pris de manière à offrir un logiciel facilement utilisable sur
    différentes plateformes mais également en vue d'aller aussi loin que
    possible dans le projet en considérant le temps imparti. Nous avons fait le
    choix de réutiliser autant que possible les outils que nous connaissions
    déjà. Cela nous a permis d'avoir une vue d'ensemble et une maîtrise que
    nous n'aurions pas eu autrement. Et nous avons pu améliorer nos
    idées mises en place pour le projet collectif de Réalité Virtuelle\refMot{bib:projet_rv}.
    
    

    \section{Liste des outils}
		\subsection{Librairie utilisée}
		
		Afin de faciliter la mise en \oe{}uvre, il est nécessaire d'utiliser des librairies.
		Nous avons donc utilisé les librairies suivantes.
		\begin{itemize}
			\item Ogre\refMot{bib:librairie_ogre}: Moteur 3D open source
			(http://www.ogre3d.org/) supportant aussi bien OpenGL que Direct3D. 
			\item CEGUI\refMot{bib:librairie_cegui}: Crazy Eddie's GUI System
			(http://www.cegui.org.uk) fournissant des outils de créations de menus
			et de fenêtres dans un environnement 3D.
			\item Bullet\refMot{bib:librairie_bullet}: Librairie de simulation de
			la physique http://bulletphysics.org/
		\end{itemize}
		
		\
		
		Notre expérience dans l'utilisation de ces librairies ayant été facilitée
		par leur utilisation dans le projet collectif\refMot{bib:projet_rv}, il
		avons grandement amélioré la structure du programme.
		
			
		\subsection{Gestionnaire de version}
		Afin de sauvegarder le projet et les données liées à celui-ci, nous avons
        choisi d'utiliser le gestionnaire de version Git. Cela nous à été
        particulièrement utile, en particulier, nous avons apprécié la gestion
        fine des conflits.
		Avec Git, il n'est pas nécessaire d'utiliser un serveur de dépôt mais
		pour une question de sauvegarde des données, nous avons utilisé le site
		gitorious.org qui propose gratuitement des dépôts.
		Le notre est stocké à l'adresse suivante : \url{https://gitorious.org/briquettephysique/briquettephysique}
		
		\subsection{Documentation technique}
		
		Il est intéressant de concevoir une documentation technique, ce qui faciliterait
		son étude par une personne tierce dans le cas d'une reprise du projet.
		
		\
		
		Nous avons utilisé pour ce faire le logiciel Doxygen\refMot{bib:outils_doxygen}.
		Ce logiciel permet à partir de commentaires spécifiques dans les fichiers d'en-tête
		de générer la documentation au format html et pdf.
		Le format html est intéressant puisqu'il permet d'être facilement diffusé sur Internet.
		
		\
		
		L'avantage majeur à utiliser Doxygen est qu'il permet la génération de graphique :
		\begin{itemize}
			\item Graphe de collaboration
			\item Graphe des dépendances par inclusion
			\item Graphe d'inclusion directe et indirecte
			\item Graphe d'appels d'une fonction
		\end{itemize}
		
        \
		
		La légende est très simple à comprendre :
		\begin{figure}[h]
			\centering
			\includegraphics[width=12cm]{images/graph_legend.png}
			\caption{\label{fig:graph_legend}Légende du graphe}
		\end{figure}
		
		\
		
		Le code permettant de générer ce graphe :
		\begin{figure}[h]
			\centering
			\lstinputlistingSmall{images/graph_legend.cpp}
			\caption{\label{fig:graph_legend_code}Le code permettant de générer le graphe de légende}
		\end{figure}
		
		
		Pour présenter les différents graphes, nous utiliserons la classe la moins utilisée dans le programme
		(GestViewport) car les graphes deviennent rapidement très grands et ils ne seraient
		pas correctement visibles dans le rapport.
		
			\subsubsection{Graphe de collaboration}
			Le graphe de collaboration de la classe GestViewport.
			\begin{figure}[h]
				\centering
				\includegraphics[width=6cm]{images/graph_collaboration_gestviewport.png}
				\caption{\label{fig:graph_collaboration_gestviewport}Graphe de collaboration de la classe GestViewport}
			\end{figure}
			
			
			\subsubsection{Graphe des dépendances par inclusion}
			Le graphe des dépendances par inclusion de la classe GestViewport.
			\begin{figure}[h]
				\centering
				\includegraphics[width=15cm]{images/graph_dependance_gestviewport.png}
				\caption{\label{fig:graph_dependance_gestviewport}Graphe des dépendances par inclusion de la classe GestViewport}
			\end{figure}
			
\newpage
			
			\subsubsection{Graphe d'inclusion directe et indirecte}
			Le graphe d'inclusion directe et indirecte de la classe GestViewport.
			\begin{figure}[h]
				\centering
				\includegraphics[width=10cm]{images/graph_inclusion_gestviewport.png}
				\caption{\label{fig:graph_inclusion_gestviewport}Graphe d'inclusion direct et indirect de la classe GestViewport}
			\end{figure}
			
			
			\subsubsection{Graphe d'appels d'une fonction}
			Le graphe d'appels de la méthode addViewport de la classe GestViewport.
			\begin{figure}[h]
				\centering
				\includegraphics[width=15cm]{images/graph_appels_addviewport_gestviewport.png}
				\caption{\label{fig:graph_appels_addviewport_gestviewport}Graphe d'appels de la méthode addViewport de la classe GestViewport}
			\end{figure}
    
    
    

\chapter{Réalisation}
    \section{Gestion des évènements claviers/souris}
        Nous utilisons une classe PlayerControl qui redirige l'ensemble des
        évènements claviers et souris en des évènements logiques. C'est cette
        classe qui permet de lier les touches à des actions, permettant
        d'abstraire la configuration des touches par la suite.

    \section{Gestion de la caméra}
        Dans ce projet, il n'y a pas besoin de fournir de multiples caméras à
        l'utilisateur. Nous avons convenu que la caméra la plus adaptée serait
        une caméra capable de pivoter autour de la table. La caméra est
        toujours orientée vers un point central, qui au départ est le centre de
        la table, cela permet d'avoir une orientation correcte, de plus l'axe
        de lacet de la caméra est fixé sur l'axe Z de façon à ce que la table
        soit toujours vue droite. Les rotations se font à l'aide de la souris,
        il est également possible de zoomer à l'aide des touches du clavier
         ou de la touche centrale de la souris (molette).
        Au départ, nous avons permis avec les touches directionnelles du clavier de
        déplacer le point central qui sert de repère à la caméra, en pratique
        ce n'est pas très utile car peu maîtrisable pour l'utilisateur. Par
        contre, un système beaucoup plus intéresant consiste à déplacer ce point
        central à la position d'une briquette en effectuant un clic droit sur
        celle-ci. La caméra se tourne alors vers la position de la briquette et les rotations
        se font autour de celle-ci.

    \begin{figure}[h]
		\centering
        \includegraphics[width=12cm]{images/bon_score.png}
        \caption{\label{fig:bon_score}Ici, on voit que la caméra n'est plus
        centrée autour du point
        central de la table mais sur une des briquettes de la structure,
        permettant d'étudier plus précisément celle-ci.}
    \end{figure}

    \section{Gestion de la physique des briquettes}
        La physique est gérée avec la librairie Bullet. La technique consiste à
        lier à l'objet d'Ogre, un 'corps' particulier à Bullet sur lequel
        s'exerceront des forces et sur lequel on surveillera les possibles
        collisions. Bullet permet de rajouter une force d'attraction globale au
        monde et un poids à chacun des objets. Une des difficultés était de
        pouvoir outre-passer la gestion de la physique offerte par bullet lors
        de la sélection des briquettes pour les replacer. Pour cela il a fallu
        désactiver l'ensemble des forces s'exercant sur l'objet, permettre les
        déplacements de l'objet (lorsqu'un objet est stable, et qu'il n'y a pas
        de risque que celui-ci subisse de nouvelles collisions, bullet le bloque
        de façon à ne pas avoir à contrôler continuellement son état), et enfin
        empêcher la mise à jour de l'objet et l'application des nouvelles
        forces avant que l'utilisateur n'ait fini de positionner l'objet. 

		\
		
        On utilise en réalité Bullet via OgreBullet pour l'intégrer avec Ogre,
        cependant OgreBullet est limité dans ses possibilités et il nous a
        parfois fallu travailler directement sur l'objet Bullet pour pouvoir
        obtenir le comportement souhaité. 
        
        \
        
        Deplus l'activation de Bullet est très simple, il suffit d'appeler la
        méthode \textit{stepSimulation} à chaque actualisation de la frame.
        Cette méthode a pour but d'effectuer toutes les actions du moteur
        physique : application de la gravité, mise à jour des forces
        sur les objets, gestion des collisions\dots{} Il nous est alors facile
        d'arrêter et de relancer le moteur physique à l'aide d'une condition
        avec un booléen.
             
        Afin de gérer ce booléen, nous avons mis en place un système de mutex
        afin d'éviter que plusieurs threads modifient la valeur de ce booléen.
        
        La méthode lock prend en paramètre un élément du type (void *) afin
        d'enregistrer quelle classe en fait l'appel. Il en est de même pour
        la méthode unlock. Et finalement la méthode de modification du booléen
        prend aussi en paramètre un pointeur sur la classe faisant l'appel
        pour savoir si elle a le droit d'effectuer la modification.
        

    \section{Sélection des briquettes}
        Pour sélectionner nos briquettes nous fournissons à l'utilisateur un
        curseur de souris. Lors du clic, un rayon est émis dans la direction de
        celui-ci, qui rapporte l'identifiant d'une briquette éventuellement
        atteinte. C'est également la librairie Bullet qui nous fournit les
        primitives nécessaires. Lorsqu'une briquette est sélectionnée, celle-ci
        prend une couleur violette pour avertir l'utilisateur.
        
    \section{Déplacement d'une briquette}
        Pour le déplacement d'une briquette, nous avons tout d'abord voulu
         utiliser la distance parcourue par la souris sur l'écran
        , puis directement déplacer la briquette en fonction de cette distance en X et Y.
        Le problème est que selon l'orientation de la caméra, la distance en X
        peut s'inverser et donc un déplacement de la souris vers la droite
        déplacerait la briquette vers la gauche.
        
        \
        
        Nous avons donc utilisé une autre méthode.
        Lors de la saisie d'une briquette, on crée un plan sur l'axe de positionnement
        des briquettes, puis nous effectuons un lancer de rayon depuis la position de
        la souris à l'écran. Ce plan possède la même largeur que celle d'une briquette.
        Le plan doit absolument être détruit après le lancer de rayon, sinon celui-ci intéragira avec
        les briquettes comme s'il y avait une collision, et les briquettes seraient alors projetées en dehors
        de leur axe commun Y.
        
        Lors du déplacement de la souris, le rayon lancé percute alors un point du plan
        , il nous suffit alors de récupérer ce point et de déplacer la briquette aux mêmes
        coordonnées en X et Z que le point de collision (Y étant l'axe commun à toutes les briquettes).
        C'est pour cela, que le centre de la face latérale de la briquette est toujours
		située à la même position que la souris.
		De plus, le plan ayant une taille limitée, il limite alors le déplacement de la briquette dans la scène.
        
        \
        
        Les briquettes tombant à côté de la table sont réinitialisées au centre
        de la table sur l'axe de construction.
		
		
		\begin{figure}[h]
			\centering
			\includegraphics[width=16cm]{images/move_briquette.png}
			\caption{\label{fig:move_briquette}On peut voir explicitement sur cette image le plan de collision
			qui est nécessaire au déplacement de la briquette.}
		\end{figure}
		
		
    

    \section{Menus}
        Le menu à été réalisé en utilisant la bibliothèque CEGUI. Une classe
        virtuelle Fenetre à été utilisée pour permettre de créer différents
        types de Fenetre. Elle permet en particulier de mettre en place des
        mécanismes en gardant la même apparence graphique pour les différentes
        fenêtres que l'on crée, en offrant des méthodes pour les créer
        simplement. Il suffit d'insérer les boutons ou les éléments graphiques
        sur une fenêtre créée automatiquement.
        
        \
        
        La charte graphique provient de deux "skins" CEGUI pré-existants et
        placés sous licence libre. Il s'agit de "TaharezLook" qui est le thème
        par défaut de CEGUI et le thème "Sleekspace" que l'on peut télécharger
        sur le site. Ce thème est élégant mais n'est ni complet ni à jour. Je
        l'ai légèrement modifié pour le faire fonctionner (par exemple les
        labels n'apparaissaient plus correctement). Les thèmes utilisent un
        document xml pour décrire leur apparence qui est appelée le "Falagard
        looknfell system". J'ai donc travaillé dans le fichier xml présent
        dans media/CEGUI/looknfeel/TaharezLook.looknfeel.
        Le logiciel dispose d'un menu d'introduction, simple qui permet de
        choisir le niveau de difficulté via 3 boutons: facile, intermédiaire,
        difficile.


		\begin{figure}[h]
			\centering
			\includegraphics[width=16cm]{images/Menus_initial.png}
			\caption{\label{fig:menus_initial}Le menu d'introduction au jeu}
		\end{figure}
		
		\
		   
        Pendant le jeu, un menu est disponible en bas de l'écran. Il est
        également très simple. Il dispose de 4 boutons, le premier affiche le
        nombre de briquettes dont l'utilisateur peut disposer (qui ne sont pas
        encore en jeu). Un clic sur ce bouton permet de faire apparaître une
        briquette sur le plateau de jeu. Le second bouton permet de suivre les
        révisions (voir la partie sur les Snapshoots). Celui-ci est composé de 2
        numéros, le premier annonce la révision actuelle sur laquelle on se
        trouve, le deuxième, le nombre de révisions accessibles. Les deux
        derniers boutons permettent respectivement d'activer ou de désactiver le
        moteur physique et d'aligner les briquettes sur l'axe de construction.
		
		\begin{figure}[h]
			\centering
			\includegraphics[width=16cm]{images/menu_jeu.png}
			\caption{\label{fig:menu_jeu}Le menu en cours de jeu}
		\end{figure}
		
		\

        Au départ, nous avions une souris gérée par Ogre (avec un overlay) qui
        permettait à l'utilisateur de viser les briquettes. CEGUI apporte sa
        propre souris. Le plus simple à été de ne conserver visible que la
        souris de CEGUI tout au long du jeu. En pratique, on a conservé de
        l'ancienne souris d'Ogre qu'un jeu de coordonnées mis à jour à chaque
        déplacement de la souris de CEGUI afin de garder la synchronisation
        entre les deux.
        

    \section{Gestion des révisions}
    
		La plus grande difficulté d'utilisation du jeu était de ne pas faire
		d'erreurs pour ne pas devoir tout recommencer. Nous avons alors créer
		un système permettant de se déplacer parmi tous les changements effectués.
		Ces changements correspondent à l'ajout d'une nouvelle briquette
		, à l'alignement des briquettes, au déplacement d'une briquette
		ou tout simplement à l'ajout d'une révision par l'utilisateur.
		
		\
		
		Ce système est composé de 3 structures, comme on peut le voir
		 sur le graphique de collaboration de la classe GestSnapShoot :
		\begin{figure}[h]
			\centering
			\includegraphics[width=8cm]{images/graph_collaboration_gestsnapshoot.png}
			\caption{\label{fig:graph_collaboration_gestsnapshoot}Le graphe de collaboration de la classe GestSnapShoot}
		\end{figure}
		
		\
		
		Maintenant, nous allons vous expliquer en détails le fonctionnement de
		ces 3 structures :
		
		\subsection{SnapShootData}

		Cette structure de données contient 3 éléments :
		\begin{itemize}
		\item Un pointeur du type ObjBriquette sur la briquette concernée par la donnée.
		\item Un vecteur de dimension 3 correspondant à la position de la briquette lors
		la capture.
		\item Un quaternion correspondant à l'orientation de la briquette.
		\end{itemize}



			\subsection{SnapShoot}

			Cette classe correspond à une révision, un état du système qui contient la position
			et l'état des briquettes affichées dans la scène.
			Elle contient donc une liste d'élément de type SnapShootData.

			\

			Lors de la création d'une nouvelle instance de la classe (appel du constructeur)
			, le moteur physique de bullet est arrêté et la liste de l'instance est complété
			avec les briquettes affichées.

			\

			Cette classe contient donc entierement l'état du jeu à un instant donné.
			Il suffit donc de cacher toutes les briquettes stockées en mémoire et de
			ré-afficher en repositionnant les briquettes contenu dans la liste.
			
			\
		
			\subsection{GestSnapShoot}
			Cette classe permet de gérer l'ensemble des révisions, elle contient
			une liste du type deque (liste "queue" accessible dans les deux sens).
			La classe permet d'ajouter une révision, d'en supprimer un ensemble depuis
			le début ou la fin, d'annuler une modification, de re-effectuer une modification
			 et surtout d'appliquer une révision donnée.
			 
        

    \section{Gestion du jeu}
    
		La plupart des actions effectuées par l'utilisateur, comme par exemple la gestion
		des révisions, n'appellent pas directement une méthode de GestSnapShoot. Elles appellent
		une méthode du même nom contenu par GestGame, cette sur-couche effectue l'action demandée
		par l'utilisateur mais elle permet de faire aux méthodes d'actualisation de l'affichage
		car au niveau structurel GestSnapShoot ne doit pas faire appel aux méthodes d'affichage.
		De plus, GestGame permet de gérer le niveau de difficulté et par conséquent le nombre maximum
		de briquettes en jeu.

    \section{Calcul du score}
        L'objectif est de construire la flèche la plus éloignée possible du bord
        de la table. Pour cela, nous avons mis en place un outil permettant de
        voir le score atteint. Durant le jeu, il suffit d'appuyer sur la touche
        "C" (Calcul) pour voir un menu donnant le score s'afficher.
        Celui-ci est calculé à partir de la briquette stable la plus éloignée
        de la table. Si la briquette vient d'être placée et qu'elle bouge encore
        un peu, elle ne sera pas comptabilisée immédiatement. Cela permet
        d'éviter d'avoir un score important en projettant une briquette à
        l'autre bout du jeu. Le calcul est fait entre le milieu de la briquette
        et le bord de la table.
    \begin{figure}[h]
		\centering
        \includegraphics[width=13cm]{images/score.png}
        \caption{\label{fig:jeux}Un exemple typique de l'application en cours de fonctionnment}
    \end{figure}
        
\chapter{Conclusion}
    La réutilisation d'outils que nous avions commencé à voir dans le projet
    collectif\refMot{bib:projet_rv} nous a permis de gagner en efficacité et de mener
    le projet à bien. En particulier nous avons implémenté correctement la
    physique du jeu. Nous avons profité de cette aisance pour offrir de
    nombreuses améliorations, aidant la prise en mains du jeu (par exemple la
    possibilité d'arrêter le moteur physique, permettant de placer plusieurs
    briquettes avant de relancer la physique).
    
    \

    Le projet pourrait être continué de plusieurs manières, on pourrait mettre
    en place un langage permettant de placer les briquettes via des scripts et
    de tester les solutions ainsi obtenues. On pourrait également travailler pour
    perfectionner la stabilité des briquettes, une solution pourrait être de
    mettre en place un système de bords collants, pour placer la briquette au
    mieux, évitant ainsi un mouvement de chute pouvant mettre en péril la
    structure.



\chapter{Bibliographie}

\begin{bibliographie}	
	\biblioentry{bib:ogre_quaternion}{} Ogre Wiki, \textit{Utilisation des quaternions dans Ogre}, \url{http://www.ogre3d.org/tikiwiki/Quaternion+and+Rotation+Primer}
	\biblioentry{bib:librairie_ogre}{} Ogre Documentation, \textit{Documentation Doxygen d'Ogre}, \url{http://www.ogre3d.org/docs/api/html/index.html}
	\biblioentry{bib:librairie_cegui}{} CEGUI Wiki, \textit{Utilisation de CEGUI}, \url{http://www.cegui.org.uk/wiki/index.php/Main_Page}
	\biblioentry{bib:librairie_bullet}{} Bullet Wiki, \textit{Utilisation de Bullet}, \url{http://bulletphysics.org/wordpress/}
	\biblioentry{bib:librairie_ogrebullet}{} Ogre Wiki, \textit{Tutorial d'utilisation d'OgreBullet}, \url{http://www.ogre3d.org/tikiwiki/OgreBullet}
	\biblioentry{bib:signaux_qt}{} Site du Zéro, \textit{Signaux avec QT}, \url{http://www.siteduzero.com/tutoriel-3-11268-les-signaux-et-les-slots.html}
	\biblioentry{bib:outils_doxygen}{} Doxygen, \textit{Génération de documentation}, \url{http://www.stack.nl/~dimitri/doxygen/}
	\biblioentry{bib:projet_rv}{} Simulation Spatiale, \textit{Projet Polytech'Tours de Réalité Virtuelle, 2011}, \url{http://code.google.com/p/rv-simulation-interactive-battle-space/}
\end{bibliographie}	


\end{document}
